% #############################################################################
% Abstract
% !TEX root = ../main.tex
% #############################################################################

\noindent
This work introduces an ultra-low noise magnetic flow cytometer for circulating tumor cell detection. The system comprised analog bias and amplification, analog-to-digital conversion, and signal processing stages. The sensor biasing circuit offers two distinct topologies: one with the sensor inside the feedback loop and the other with the sensor outside that allow precise 16-bit digital control of the current, delivering up to $\mathrm{5~mA}$ for a sensor with a nominal resistance of $\mathrm{500~\Omega}$. The two-stage amplification utilizes differential AC-coupled inputs providing a total gain of $\mathrm{80~dB}$ over two stages. A $\mathrm{1~nV/\sqrt{Hz}}$ ultra-low noise instrumentation amplifier is followed by a non-inverting amplifier topology, while a detection circuit provides a visual and digital alert of possible first-stage saturation. A 4\textsuperscript{th}-order anti-aliasing filter, provides a bandwidth limitation of $\mathrm{100~kHz}$. The signal chain achieves a noise level of $\mathrm{4~nV/\sqrt{Hz}}$ at a frequency of $\mathrm{10~kHz}$. That to the best of the author's knowledge surpasses any previously reported work. The system incorporates a sensor 24 to 6 multiplexing scheme, enabling the digital selection of the desired sensors. The 6 analog channels are multiplexed to two high-speed $\mathrm{65~MSPS}$ ADCs connected to a DE-10 Standard FPGA for digital signal processing. The board can be powered by batteries, USB, or FPGA. The functionalities of the platform have been successfully implemented and validated.
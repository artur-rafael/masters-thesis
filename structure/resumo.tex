% #############################################################################
% Resumo
% !TEX root = ../main.tex
% #############################################################################

\noindent
Este trabalho apresenta um citómetro de fluxo magnético de ultra baixo ruído para deteção de células tumorais circulantes. O sistema é composto por etapas de bias e amplificação analógica, conversão analógico-digital e processamento de sinal. O circuito de polarização do sensor oferece duas topologias distintas: uma com o sensor dentro da malha de realimentação e outra com o sensor fora, permitindo um controlo digital preciso de 16-bits da corrente até $\mathrm{5mA}$ para um sensor com uma resistência nominal de $\mathrm{500\Omega}$. A amplificação de dois estágios utiliza entradas diferenciais com acoplamento AC, proporcionando um ganho total de $\mathrm{80dB}$ nos dois estágios. Um amplificador de instrumentação de ultra baixo ruído de $\mathrm{1nV/\sqrt{Hz}}$ é seguido por uma topologia de amplificador não inversor, enquanto um circuito de deteção fornece um alerta visual e digital de possível saturação no primeiro estágio. Um filtro antialiasing de 4ª ordem limita a largura de banda a $\mathrm{100kHz}$. A cadeia de sinal alcança um nível de ruído de $\mathrm{4nV/\sqrt{Hz}}$ a uma frequência de $\mathrm{10kHz}$, que, segundo o conhecimento do autor, supera qualquer trabalho previamente relatado. O sistema incorpora um esquema de multiplexagem de 24 sensores para 6, permitindo a seleção digital dos sensores desejados. Os 6 canais analógicos são multiplexados por dois ADCs de alta velocidade de $\mathrm{65MSPS}$ conectados a uma FPGA DE-10 Standard para processamento digital de sinal. A placa pode ser alimentada por baterias, USB ou FPGA. As funcionalidades da plataforma foram implementadas e validadas com sucesso.
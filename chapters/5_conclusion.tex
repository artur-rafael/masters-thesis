% #############################################################################
% Chapter 4 - Conclusion
% !TEX root = ../main.tex
% #############################################################################

%\fancychapter{Conclusion}
\newfancychapter{Conclusion}{
This chapter offers a concise and formal overview of the research conducted. It includes a comprehensive review of the main findings and important observations made throughout the study. Critical remarks are provided, addressing any limitations and challenges encountered. Additionally, the future work section outlines potential areas for further exploration and improvement, providing a roadmap for future studies. Overall, this chapter serves as a conclusive summary, offering valuable insights and guiding future advancements in the field.
}
\label{chapter:conclusion}

% Uncomment before printing
\clearpage
%\thispagestyle{empty}
%\cleardoublepage

% ############################################################################# Review and Remarks
\section{Review and Remarks}
\label{section:conclusion-conclusion}

The research described in this dissertation consists of an discrete electronics system for interfacing \ac{MR} sensors, featuring a fully analog path, from the sensor bias to digital conversion. However, the new platform has to consider the specifications defined for the early cancer diagnosis application. The noise generated within the analog path is propagated to the output, degrading the overall quality of the measurement. Thus, it was required to develop an ultra-low noise state-of-the-art interface. The cytometer platform conceived in this dissertation enriched an existing \ac{MFC} prototype developed in \ac{INESC}.

Significant changes were introduced to the biasing architecture, impacting various aspects, including the reference voltage used throughout the system. The ability to digitally and effortlessly adjust the reference voltage was incorporated into the design, providing flexibility and control. Furthermore, the biasing topology was carefully engineered to seamlessly switch between two distinct configurations. These configurations differ in the placement of the sensor relative to the feedback loop. One configuration involves positioning the sensor inside the feedback loop, while the other maintains the sensor outside the loop. The intention behind implementing two different topologies was to explore potential improvements in reducing noise introduced by the biasing architecture. However, as this represents a novel approach, the effectiveness of the sensor placement within the feedback loop has yet to be substantiated by empirical evidence. In the event that the new topology fails to meet expectations, the previous configuration, with the sensor positioned outside the feedback loop, can serve as a fallback option. This alternative ensures the continuity of the system's functionality and mitigates risks associated with unproven methodologies. Additionally, apart from these significant changes, the architecture also tackled noise-related concerns through the utilization of an alternative reference voltage generator. This modification aimed to further enhance the system's performance and minimize noise interference.

The amplification scheme, which played a crucial role in enhancing the signal response of the sensors, remained unchanged in terms of the circuit design. However, certain modifications were made to incorporate specific switches that facilitate noise measurement and enable potential tests within the circuit. This enhancement further contributes to the overall understanding and refinement of the amplification scheme's performance within the system. In addition to these changes, the saturation detector circuit of the first amplification stage has been improved. This fix allows for not only computer-based detection but also visual identification of saturation, regardless of the duration of the saturation event. By incorporating this fix, the means of detecting saturation have expanded, providing a comprehensive approach to monitor and address potential signal saturation issues.

A new circuit was developed to cater to the majority of sensors within the sensor chip. This enables each of the 6 analog channels to select at least 4 distinct sensors. This capability is particularly valuable since the chip houses a total of 28 sensors, organized across 8 microfluidic channels. While this new feature is undoubtedly important and beneficial, it does come with the minor drawback of requiring an additional circuitry module to control the multiplexers used for this purpose. The introduction of this new circuit represents a significant advancement in sensor integration and utilization. By expanding the selection options for each analog channel, the system gains versatility in addressing different sensors within the chip. This flexibility allows for a more comprehensive analysis of various samples, enhancing the overall capabilities of the platform. 

In order to seamlessly integrate the interface within the larger \ac{MFC} system, supplementary circuits were developed to accommodate the limitations of the subsequent systems, particularly the \ac{FPGA}. As a result, careful considerations were made regarding the digital path of the interface. To overcome the limitations imposed by the number of available \ac{ADC}s and their input voltage range, a gain reduction multiplexing circuit was implemented. This module was specifically designed with the intention that the bottleneck lies on the \ac{FPGA} side rather than the interface itself. By implementing this circuit, the interface can effectively optimize the usage of \ac{ADC}s, ensuring efficient utilization of resources and maximizing the system's overall performance. Additionally, to facilitate compatibility with not only the \ac{FPGA} but also other potential control systems, level shifters were employed for all the required inputs and outputs of the interface. This enables seamless communication and interaction between the interface and various external systems, ensuring compatibility and ease of integration.

In response to the heightened demand for supply current resulting from the increased number of analog channels, the power supply configuration of the board underwent a thorough redesign. This redesign aimed to effectively accommodate the higher power requirements and ensure optimal performance. In addition to addressing the increased demand, the redesign also aimed to enhance the flexibility of the board by considering new supply sources. The board now offers expanded options for powering, including the \ac{FPGA} or a USB connection as a power source. These additions provide greater versatility and convenience in powering the board, alongside the previous power source from batteries. By incorporating these improvements to the power supply configuration, the board gains enhanced flexibility, ensuring compatibility with different power sources and providing users with more options for powering the system. This optimization supports efficient operation and contributes to the overall usability of the board in diverse settings.

In conclusion, the research presented in this dissertation has made significant contributions to the development of a robust and advanced discrete electronics system for interfacing \ac{MR} sensors. The proposed improvements and advancements in circuit design, biasing architecture, amplification scheme, sensor integration, digital system integration, and power supply configuration lay the foundation for further advancements in  the \ac{MFC} system. The comprehensive understanding gained through this document enables to delve deeper into the intricacies of \ac{MR} sensor interfacing and leverage the advancements made to drive further breakthroughs in early cancer diagnosis and related applications.
% ############################################################################# Review and Remarks

% ############################################################################# Future Work
\section{Future Work}
\label{section:conclusion-future}

This section outlines potential future work that can build upon the achievements and advancements made in this thesis. While significant progress has been made in developing a robust electronics system for interfacing \ac{MR} sensors, there are still opportunities for further exploration and improvement. The following paragraphs provide an overview of potential areas for future research and development, including interface tests, application tests, and future interface versions.

A specific focus on testing the "closed-loop" bias topology and the new supply features, it is essential to highlight the significance of evaluating their noise performance. By conducting thorough tests and analyses, researchers can gain insights into the effectiveness of these features in reducing noise interference. The biasing topology with the sensor outside the biasing feedback loop allows the op-amp to mitigate some of the noise the topology introduces. Additionally, synchronizing the acquisition clock signal with the \ac{SEPIC} and inverting circuit can contribute to minimizing the impact of ripple noise on the system. These areas warrant further investigation to optimize the noise characteristics of the system and ensure its robust performance in practical applications.

In addition to conducting interface tests, it is crucial to carry out real experiments involving magnetic fields generated by magnetic nanoparticles. These experiments will serve as concrete evidence of the effectiveness of the developed interface in fulfilling its intended purpose. Furthermore, the results obtained can be compared with those obtained using the previous platform, allowing for an evaluation of the overall benefits of the new system. Additionally, it is essential to integrate the findings and developments from other theses conducted concurrently with this work. This integration will provide a comprehensive assessment of the desired \ac{MFC} system and enable a thorough understanding of its potential advantages and contributions.

In future iterations of the cytometer platform, the anticipated resolution of the semiconductor shortage opens up opportunities for improvement. The selection of alternative multiplexers that align with the interface objectives becomes crucial. The current multiplexers used in the sensor addressing circuit introduce unnecessary complexity due to their communication protocol and the excessive number of inputs. Additionally, the power rail requirements of the multiplexers in the output multiplexing circuit differ from other \ac{IC}s, necessitating the generation of two additional power rails. Therefore, exploring alternative multiplexer options is essential for future platform versions. Despite the absence of space constraints, the developed interface remains relatively large. However, the modular design allows for a seamless transition to a more compact system if desired. Partitioning the current platform into multiple boards, like the previous platform, enables flexibility and scalability. Another idea for reducing space further is to implement analog channels on small \ac{PCB}s with gold fingers, facilitating their insertion similar to a memory stick in a computer.

% ############################################################################# Future Work

\clearpage
\thispagestyle{empty}
\cleardoublepage